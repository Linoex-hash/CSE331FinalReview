\documentclass{beamer}
\usepackage{dirtytalk}
\usetheme{Frankfurt}
\title{CSE 331 Final Exam Preparation}
\subtitle{This is in no way a substitute for exam preperation, mearly a compilation of all the key talking points.}
\author{Chandra Neppalli}
\date{\today}
\begin{document}
    \maketitle
    \section{Basic Proof Techniques}
    \begin{frame}{Counter Example}  
        \onslide<1->Best proof to use to \textit{disprove} universally true propositions.
        \bigskip

        \onslide<2->Ex: Every day is a Wednesday, where a counter example would be Monday is not Wednesday.
    \end{frame}
    \begin{frame}{Contradiction}
        \onslide<1->Best proof to use if you want to assert something is true.
        \bigskip

        \onslide<2->Assume what you want to prove is false, then show this leads to a contradiction.
        \bigskip

        \onslide<3->Therefore, the original assumption has to be true.            
    \end{frame}
    \begin{frame}{Contraposition}
        \onslide<1->Best proof for proving causality. Define two propositions E and F.
        \bigskip

        \onslide<2->{If you want to prove that E $\rightarrow$ F, 
        it might be more doable to prove $\lnot$F $\rightarrow$ $\lnot$E, 
        as they are both logically equivalent.}
        \bigskip

        \onslide<3->This is especially useful if the \textbf{scope} of F is smaller than the scope of E.
    \end{frame}
    \begin{frame}{Direct Proof}
    \label{frame:directproof}
        \onslide<1->If the proof is \textit{simple}, consider directly proving it.
        \bigskip

        \onslide<2->{Remember though, that you must maintain $W.L.O.G$, 
        that your proof can never be too specific and must be arbitrary.}   
    \end{frame}
    \begin{frame}{Proof by Induction}
        \onslide<1->{Proof by Induction is a really nice proof technique when you reduce your proof to a known 
        correct base case.}
        \bigskip

        \onslide<2->{If proof needs to be correct for all numbers $\in \mathbb{N}$, and each step is dependant on the previous step, then \textit{every}
        step can be reduced to a definitive base case that is easy to \hyperlink{frame:directproof}{directly prove}.}
    \end{frame}
    \begin{frame}{Extra: Progress Measure}
        \onslide<1->This is useful for proving an algorithm with a loop terminates.
        \bigskip

        \onslide<2->Let $P(i)$ denote an integer such that:
        \begin{itemize}
            \item<3-> $P(0) = l$ 
            \item<4-> $P(i)$ is an accumulator. This means that $P(i + 1) > P(i)$
            \item<5-> $\forall i,\;\; P(i) \leq k$
        \end{itemize}
        \onslide<6->From these 3 properties, the number of iterations is bounded by $k - l + 1$
        \bigskip

        \onslide<7->Note: This isn't a runtime analysis, rather a proof that the algorithm terminates.
    \end{frame}
    \begin{frame}{Greedy Stays Ahead}
        \onslide<1->This technique is used to prove that a greedy algorithm returns an optimal solution.
        \bigskip

        \onslide<2->At every step of a greedy algorithm, it will stay \textit{at least} as far as the optimal solution at that step.
        \bigskip

        \onslide<3->HW4 \say{Attack on Alarms} and Interval Scheduling are examples of problems with greedy solutions.
    \end{frame}
    \section{The Stable Matching Problem}
    \begin{frame}{Introduction}
        \onslide<1->Let's say there are two groups: Group A and Group B.
        \bigskip

        \onslide<2->How do we generate a \textbf{stable} matching between each member of the two groups efficiently?
        \bigskip

        \onslide<3->Moreover, what is a \textbf{stable} matching?
    \end{frame}
    \begin{frame}{Perfect Matchings}
        \onslide<1->A \textbf{perfect matching} is a bijective matching between A and B.
        \bigskip

        \onslide<2->\textbf{Every member} in group A is matched with exactly one member in group B.
        \bigskip

        \onslide<3->Conversely, every member in group B is matched with \textbf{exactly} one member in group A.
        \bigskip

        \onslide<4->With $n$ members in each group, there are $n!$ perfect matchings.
    \end{frame}
    \begin{frame}{Instability}
        \onslide<1->For a particular matching, define a member $m$ from group A and $n$ from group B such that ($m,n$) is not in the matching.
        \bigskip

        \onslide<2->If $m$ prefers $n$ over their current matching \textbf{and} $n$ prefers $m$ over their current matching, the entire match is an instability.
    \end{frame}
    \begin{frame}[fragile]{Stable Matching}
        \only<-5>{
            \onslide<1->A stable matching \textbf{is} a perfect matching with no instability.
            \bigskip

            \onslide<2->It therefore follows that the number of stable matchings 
            is \textit{at least} the number of perfect matchings, or $n!$
            \bigskip

            \onslide<3->The Gale Shapely Algorithm is an $O(n^3)$ time algorithm that can output a stable matching.
            \bigskip

            \onslide<4->With the right data structures, the runtime can be reduced to $O(n^2)$. 
            \bigskip

            \onslide<5->Even though the runtime isn't linear, because the input size is $2n^2 \rightarrow \Theta(n^2)$\footnote<5->{This comes from $n$ Group A members and $n$ Group B members with their $2n$ preference lists}, the runtime \textbf{with respect} to the input size is $O(N)$, or linear time.
        }
        \only<6>{
            {\small Code}:
            \rule{\textwidth}{0.5pt}
            \texttt{\footnotesize Initially all $m \in M$ and $w \in W$ are free\\
                While there is a man $m$ who is free and hasn’t proposed to
                every woman\\
                \hspace{1cm}Choose such a man $m$\\
                \hspace{1cm}Let $w$ be the highest-ranked woman in $m$’s\\
                \hspace{2cm}preference list to whom $m$ has not yet proposed\\
                \hspace{1cm}If $w$ is free then\\
                \hspace{2cm}$(m, w)$ become engaged\\
                \hspace{1cm}Else $w$ is currently engaged to $m'$\\
                \hspace{2cm}If $w$ prefers $m'$ to $m$ then\\
                \hspace{3cm}$m$ remains free\\
                \hspace{2cm}If $w$ prefers $m$ to $m'$\\
                \hspace{3cm}$(m, w)$ become engaged\\
                \hspace{3cm}$m'$ becomes free\\
                \hspace{2cm}Endif\\
                \hspace{1cm}Endif\\
                EndWhile\\
                Return the set $S$ of engaged pairs.
            }
        }
    \end{frame}    
\end{document}